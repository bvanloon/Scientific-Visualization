%!TEX root = ../scivis_lbaakman_bvanloon.tex
In this section the main methods are given that are used to construct the colormaps. Furthermore 
\todo[inline]{Introduce the section}



\subsection{Parameterization of Color Maps}
\label{ss:colormaps:parameterization}
	\todo[inline]{Number of colors, disadvantages/advantages when to use?}
	\todo[inline]{Change hue, disadvantages/advantages when to use?}
	\todo[inline]{Change saturation, disadvantages/advantages when to use?}

\subsection{Applying Color Maps}
\label{ss:colormaps:applying}
	\todo[inline]{Discuss clamping}
	\todo[inline]{Discuss scaling}

\subsection{Variables that the Color Maps are Applied to}
\label{ss:colormaps:variables}
	\todo[inline]{Discuss fluid density, what is it? suitable for colormapping? Which colormap?}
	\todo[inline]{Discuss fluid velocity magnitude what is it? suitable for colormapping? Which colormap?}
	\todo[inline]{Discuss force field magnitude what is it? suitable for colormapping? Which colormap?}