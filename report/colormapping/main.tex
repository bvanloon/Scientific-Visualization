%!TEX root = ../scivis_lbaakman_bvanloon.tex

\chapter{Color Mapping} % (fold)
\label{cha:color_mapping}

Color mapping is widely used for the visualization of scalar data. This is done by associating a color with each scalar value that is visualized. In this chapter we will discuss the colormaps included in the application and how these colormaps are supported. In \cref{s:colormapping:method} the general methology used for colormapping is first explained. Next \Cref{s:colormaps:differentmaps} highlights the important features of colormaps and discusses the colormaps that are implemented. For each colormap its advantages and disadvantages are given, as well as a small motivation why it is included in the application. Finally, \cref{s:colormapping:results} shows the results of applying colormaps to the simulation and discusses these results.

% In the next section, \Cref{s:colormaps:differentmaps}, several colormaps are discussed that are implemented in the application.  In \cref{ss:colormaps:parameterization} the parameterization of the color maps is introduced and \Cref{ss:colormaps:applying} discusses techniques that can be used to map the scalar data to the color maps. In the final section, \cref{ss:colormaps:variables}, the application of the color maps to the scalar variables in our application is discussed. 

\section{Method}
\label{s:colormapping:method}
%!TEX root = ../../scivis_lbaakman_bvanloon.tex
\section{Method}
\label{s:streamsurfaces:method}
\todo[inline]{Introduction into the section}

\todo[inline]{Refer to the streamlines}

\todo[inline]{How do we build a surface}

\todo[inline]{Divergence sensitivity}

\section{Provided Color Maps}
\label{s:colormaps:differentmaps}
%!TEX root = ../scivis_lbaakman_bvanloon.tex
This section discusses the colormaps provided in the application. Each colormap is accompanied with a short discussion giving the main goal of the colormap and its pros and cons.

Colormaps are used to map scalar values to colors in order for the user to extract information about the scalar values in the visualization. The quality of a colormap can be judged by how well and intuitive the map allows inverse mapping. Which colormap is best used depends on the context and goal of the visualization. The zebra-colormap is for example well suited to visualize value variations (i.e. the first derivative of a scalar value), but would be less useful would the aim be to highlight maxima. Historical reasons can also influence the choice of colormaps. X-rays are for example standard displayed using a gray-scale colormap. Better alternatives might exist, but are not used due to status quo in the medical field.


\begin{figure}
	\centering
	\begin{subfigure}{0.44\textwidth}
		\centering
		\includegraphics[width=\textwidth, trim={35px 30px 430px 30px}, clip]{colormapping/img/rainbow.png}
		\caption{Rainbow colormap, using $\VarDX = 0.8$.}
		\label{fig:colormapping:intro:differntColorMaps:rainbow}
	\end{subfigure}
	\begin{subfigure}{0.44\textwidth}
		\centering
		\includegraphics[width=\textwidth, trim={35px 30px 430px 30px}, clip]{colormapping/img/zebra}
		\caption{Zebra colormap showing the first derivative.}
		\label{fig:colormapping:intro:differntColorMaps:zebra}
	\end{subfigure}	
	\begin{subfigure}{0.44\textwidth}
		\centering
		\includegraphics[width=\textwidth, trim={35px 30px 430px 30px}, clip]{colormapping/img/grayscale}
		\caption{Gray-scale colormap.}
		\label{fig:colormapping:intro:differntColorMaps:grayscale}
	\end{subfigure}	
	\begin{subfigure}{0.44\textwidth}
		\centering
		\includegraphics[width=\textwidth, trim={35px 30px 430px 30px}, clip]{colormapping/img/heat.png}
		\caption{Heat colormap.}
		\label{fig:colormapping:intro:differntColorMaps:heat}
	\end{subfigure}
	\begin{subfigure}{0.44\textwidth}
		\centering
		\includegraphics[width=\textwidth, trim={35px 30px 430px 30px}, clip]{colormapping/img/cold.png}
		\caption{Cold colormap.}
		\label{fig:colormapping:intro:differntColorMaps:cold}
	\end{subfigure}
		\begin{subfigure}{0.44\textwidth}
		\centering
		\includegraphics[width=\textwidth, trim={35px 30px 430px 30px}, clip]{colormapping/img/hue.png}
		\caption{Hue colormap using pink hue selected by user.}
		\label{fig:colormapping:intro:differntColorMaps:hue}
	\end{subfigure}
	\begin{subfigure}{0.44\textwidth}
		\centering
		\includegraphics[width=\textwidth, trim={35px 30px 430px 30px}, clip]{colormapping/img/twocolor}
		\caption{Two-color colormap.}
		\label{fig:colormapping:intro:differntColorMaps:twocolor}
	\end{subfigure}	\begin{subfigure}{0.44\textwidth}
		\centering
		\includegraphics[width=\textwidth, trim={35px 30px 430px 30px}, clip]{colormapping/img/diverging}
		\caption{Blue red diverging colormap.}
		\label{fig:colormapping:intro:differntColorMaps:diverging}
	\end{subfigure}				

	\caption{A visualization of the \emph{Fluid Density} using different colormaps available in the application. All colormaps the full range of availab}
	\label{fig:colormapping:colormaps}
\end{figure}


\subsection{Rainbow colormap} % (fold)
\label{ssub:rainbow_colormap}
The rainbow colormap is, despite its flaws, one of the most used colormaps in the scientific community. The colormap is often seen as visual appealing and gives an intuitively mapping of high and low values to warm (red) and cold (blue) colors. However, when the scalar value does not represent a temperature this mapping might be less intuitive which makes the colormap harder to use.

\subsubsection{Flaws} % (fold)
\label{ssub:flaws}
The rainbow colormap is not only often used, but also often critiqued\cite{borland2007rainbow}\cite{divergingMoreland2009}. Besides the sometimes unintuitive associated of high and low values with warm and cold colors the colormap has some more (serious) flaws. 


% subsubsection flaws (end)



\subsection{Gray Scale colormap} % (fold)
\label{ssub:gray_scale_colormap}
	% GrayScale ColorMap
	\todo[inline]{Discuss GrayScale color map: advantages, disadvantages}

Disadvantages: surface shading, same luminance may look different depending on the surrounding pixels.
% subsubsection gray_scale_colormap (end)
	% ColorMap of Choice
	\todo[inline]{Discuss Chosen color map: advantages, disadvantages, why this one}

\subsection{Heat, Cold, and Hue Colormap} % (fold)
\label{sub:heat_and_cold_colormap}

% subsection heat_and_cold_colormap (end)
\subsection{Diverging Colormap} % (fold)
\label{sub:diverging_colormap}

% subsection diverging_colormap (end)

\subsection{Zebra Colormap} % (fold)
\label{sub:zebra_colormap}

% subsection zebra_colormap (end)
% section color_maps (end)

\section{Results}
\label{s:colormapping:results}
%!TEX root = ../scivis_lbaakman_bvanloon.tex
\begin{figure}[h!]
	\centering
	\begin{subfigure}{0.35\textwidth}
		\centering
		\includegraphics[width=0.9\textwidth, trim={35px 30px 430px 30px}, clip]{colormapping/img/rainbow}
		\begin{tikzpicture}
		    \node[anchor=south west,inner sep=0] (image) at (0,0) {\includegraphics[rotate=90,width=0.03\textwidth,height=95px,keepaspectratio=false,frame]{colormapping/img/colormap_legends/rainbowcolormap}};
		\end{tikzpicture}
		\caption{Rainbow}
		\label{fig:colormapping:intro:differntColorMaps:rainbow}
	\end{subfigure}
	\hspace{30px}
	\begin{subfigure}{0.35\textwidth}
		\centering
		\includegraphics[width=0.9\textwidth, trim={35px 30px 430px 30px}, clip]{colormapping/img/zebra_166}
		\begin{tikzpicture}
		    \node[anchor=south west,inner sep=0] (image) at (0,0) {\includegraphics[rotate=90,width=0.03\textwidth,height=95px,keepaspectratio=false,frame]{colormapping/img/colormap_legends/zebracolormap}};
		\end{tikzpicture}
		\caption{Zebra}
		\label{fig:colormapping:intro:differntColorMaps:zebra}
	\end{subfigure}
	\begin{subfigure}{0.35\textwidth}
		\centering
		\includegraphics[width=0.9\textwidth, trim={35px 30px 430px 30px}, clip]{colormapping/img/grayscale}
		\begin{tikzpicture}
		    \node[anchor=south west,inner sep=0] (image) at (0,0) {\includegraphics[rotate=90,width=0.03\textwidth,height=95px,keepaspectratio=false,frame]{colormapping/img/colormap_legends/grayscalecolormap}};
		\end{tikzpicture}
		\caption{
		Gray-scale.
		}
		\label{fig:colormapping:intro:differntColorMaps:grayscale}
	\end{subfigure}
		\hspace{30px}
	\begin{subfigure}{0.35\textwidth}
		\centering
		\includegraphics[width=0.9\textwidth, trim={35px 30px 430px 30px}, clip]{colormapping/img/hue}
		\begin{tikzpicture}
		    \node[anchor=south west,inner sep=0] (image) at (0,0) {\includegraphics[rotate=90,width=0.03\textwidth,height=95px,keepaspectratio=false,frame]{colormapping/img/colormap_legends/huecolormap}};
		\end{tikzpicture}
		\caption{
		Hue (Pink)
		}
		\label{fig:colormapping:intro:differntColorMaps:hue}
	\end{subfigure}
	\begin{subfigure}{0.35\textwidth}
		\centering
		\includegraphics[width=0.9\textwidth, trim={35px 30px 430px 30px}, clip]{colormapping/img/heat}
		\begin{tikzpicture}
		    \node[anchor=south west,inner sep=0] (image) at (0,0) {\includegraphics[rotate=90,width=0.03\textwidth,height=95px,keepaspectratio=false,frame]{colormapping/img/colormap_legends/heatcolormap}};
		\end{tikzpicture}
		\caption{
		Heat
		}
		\label{fig:colormapping:intro:differntColorMaps:heat}
	\end{subfigure}
		\hspace{30px}
	\begin{subfigure}{0.35\textwidth}
		\centering
		\includegraphics[width=0.9\textwidth, trim={35px 30px 430px 30px}, clip]{colormapping/img/cold}
				\begin{tikzpicture}
		    \node[anchor=south west,inner sep=0] (image) at (0,0) {\includegraphics[rotate=90,width=0.03\textwidth,height=95px,keepaspectratio=false,frame]{colormapping/img/colormap_legends/coldcolormap}};
		\end{tikzpicture}
		\caption{
		Cold
		}
		\label{fig:colormapping:intro:differntColorMaps:cold}
	\end{subfigure}	
	\begin{subfigure}{0.35\textwidth}
		\centering
		\includegraphics[width=0.9\textwidth, trim={35px 30px 430px 30px}, clip]{colormapping/img/twocolors}
		\begin{tikzpicture}
		    \node[anchor=south west,inner sep=0] (image) at (0,0) {\includegraphics[rotate=90,width=0.03\textwidth,height=95px,keepaspectratio=false,frame]{colormapping/img/colormap_legends/twocolorscolormap}};
		\end{tikzpicture}
		\caption{
		Isoluminant Blue-Red
		}
		\label{fig:colormapping:intro:differntColorMaps:twocolor}
	\end{subfigure}	
	\hspace{30px}
	\begin{subfigure}{0.35\textwidth}
		\centering
		\includegraphics[width=0.9\textwidth, trim={35px 30px 430px 30px}, clip]{colormapping/img/diverging}
		\begin{tikzpicture}
		    \node[anchor=south west,inner sep=0] (image) at (0,0) {\includegraphics[rotate=90,width=0.03\textwidth,height=95px,keepaspectratio=false,frame]{colormapping/img/colormap_legends/divergingcolormap}};
		\end{tikzpicture}
		\caption{
		Diverging
		}
		\label{fig:colormapping:intro:differntColorMaps:diverging}
	\end{subfigure}		
\caption{A visualization of the fluid density (\density) using the color maps available in the application. All color maps uses 256 colors except for the zebra color map which uses 50. All color maps are fully saturated.}
\label{fig:colormapping:colormaps}
\end{figure}


In \cref{fig:colormapping:colormaps} a visualization from simulation snapshot is shown using all the different color maps available in the application. Comparing these figures we observe firstly that in the visualization with the rainbow color map, \cref{fig:colormapping:intro:differntColorMaps:rainbow}, the maximum values are very prominent and that there is a clear distinction between the blue, green, and green areas but no clear transition between those areas. This fits with the disadvantages discussed in \cref{ssub:rainbow_colormap}. \Cref{fig:colormapping:intro:differntColorMaps:zebra} indeed illustrates the areas with high and low variation. 

Comparing the luminance based color maps, \crefrange{fig:colormapping:intro:differntColorMaps:grayscale}{fig:colormapping:intro:differntColorMaps:cold}, we observe in all cases  a nice transition and from areas with low to areas with high scalar values. Also note the added benefit of the maps containing hues compared to the gray-scale color map; especially in the areas containing low values a bit more variation is visible.


The isoluminant blue-red color map, shown in \cref{fig:colormapping:intro:differntColorMaps:twocolor}, does not offer as much details as the other color maps. This confirms that these color maps are not well suited for 2D visualizations.

In the diverging color map, illustrated in \cref{fig:colormapping:intro:differntColorMaps:diverging}, one easily recognizes the same distinct areas shown by the rainbow color map.  Compared to the rainbow color map, there are two distinct differences. First the maximum values do not draw as much attention as they do in the rainbow color map, furthermore we can see a more natural transition from low to high values making this color map more suitable for visualization of these data.

% chapter color_mapping (end)