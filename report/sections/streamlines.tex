%!TEX root = ../scivis_lbaakman_bvanloon.tex
\chapter{Streamlines} % (fold)
\label{cha:streamlines}

A streamline is the curved path of an imaginary particle that is released at some start location, a seed point, in a stationary vector field\cite{telea2014data}. Where glyphs give a general overview of the flow within the vector field, a streamline visualize the localized aspect. Multiple streamlines can of course be used to generate a more general overview of the flow within the vector field. A disadvantage of streamlines compared to glyphs is that they have no direction. 

\Cref{s:streamlines:method} discuss the computation of streamlines, in \cref{s:streamlines:design} the considerations that guided our design of streamlines are presented. And finally \cref{s:streamlines:results} shows and discusses our results.

%!TEX root = ../../scivis_lbaakman_bvanloon.tex
\section{Method}
\label{s:streamsurfaces:method}
\todo[inline]{Introduction into the section}

\todo[inline]{Refer to the streamlines}

\todo[inline]{How do we build a surface}

\todo[inline]{Divergence sensitivity}

%!TEX root = ../../scivis_lbaakman_bvanloon.tex
\section{Design}
\label{s:streamlines:design}
The visualization of the vector field with streamlines depends on the placement of the seed points, discussed in \cref{ss:streamlines:design:seedpoints} and the parameterization of the streamline computation, discussed in \cref{ss:streamlines:design:parameterization}.

\subsection{Seed Points}
\label{ss:streamlines:design:seedpoints}
We have implemented two mechanisms of placing seed points, that can be combined. We allow the user to place the seed points manually. This allows use to use the insight of the user to gain coverage, uniformity and continuity. A disadvantage of placing seed points manually is that it is a lot of work to place seed points like this. Therefore we also offer the probability of placing a grid of seed points over the complete domain. The advantage is that it is fast. Furthermore a fine mazed grid, the grid dimensions can be set by the user, allows for good coverage. However uniformity is definitely not ensured in this way, however this can be solved by letter the user manually add seed points to the grid of seed points near areas where the density of the streamlines is low.

\subsection{Visualization Parameterization}
\label{ss:streamlines:design:parameterization}
The user can parameterize the visualization with a number of variables. The \t{time step}  and \t{maximum time} directly map to \integrationTime and \timeInterval. The length of the edges of the streamline and the upper limit of the length of the streamline are controlled by \t{edge length} and \t{maximum length}. Both of these are expressed in the width, or equivalently since our simulation cells are squares, in the height, of the simulation cells. This gives the user more insight in the length of the (elements of the) streamlines, than arbitrary numbers would have been. Lastly the user can also set the \t{minimum magnitude} to control the termination of the integration of the streamline.

%!TEX root = ../scivis_lbaakman_bvanloon.tex
\begin{figure}[h!]
	\centering
	\begin{subfigure}{0.35\textwidth}
		\centering
		\includegraphics[width=0.9\textwidth, trim={35px 30px 430px 30px}, clip]{colormapping/img/rainbow}
		\begin{tikzpicture}
		    \node[anchor=south west,inner sep=0] (image) at (0,0) {\includegraphics[rotate=90,width=0.03\textwidth,height=95px,keepaspectratio=false,frame]{colormapping/img/colormap_legends/rainbowcolormap}};
		\end{tikzpicture}
		\caption{Rainbow}
		\label{fig:colormapping:intro:differntColorMaps:rainbow}
	\end{subfigure}
	\hspace{30px}
	\begin{subfigure}{0.35\textwidth}
		\centering
		\includegraphics[width=0.9\textwidth, trim={35px 30px 430px 30px}, clip]{colormapping/img/zebra_166}
		\begin{tikzpicture}
		    \node[anchor=south west,inner sep=0] (image) at (0,0) {\includegraphics[rotate=90,width=0.03\textwidth,height=95px,keepaspectratio=false,frame]{colormapping/img/colormap_legends/zebracolormap}};
		\end{tikzpicture}
		\caption{Zebra}
		\label{fig:colormapping:intro:differntColorMaps:zebra}
	\end{subfigure}
	\begin{subfigure}{0.35\textwidth}
		\centering
		\includegraphics[width=0.9\textwidth, trim={35px 30px 430px 30px}, clip]{colormapping/img/grayscale}
		\begin{tikzpicture}
		    \node[anchor=south west,inner sep=0] (image) at (0,0) {\includegraphics[rotate=90,width=0.03\textwidth,height=95px,keepaspectratio=false,frame]{colormapping/img/colormap_legends/grayscalecolormap}};
		\end{tikzpicture}
		\caption{
		Gray-scale.
		}
		\label{fig:colormapping:intro:differntColorMaps:grayscale}
	\end{subfigure}
		\hspace{30px}
	\begin{subfigure}{0.35\textwidth}
		\centering
		\includegraphics[width=0.9\textwidth, trim={35px 30px 430px 30px}, clip]{colormapping/img/hue}
		\begin{tikzpicture}
		    \node[anchor=south west,inner sep=0] (image) at (0,0) {\includegraphics[rotate=90,width=0.03\textwidth,height=95px,keepaspectratio=false,frame]{colormapping/img/colormap_legends/huecolormap}};
		\end{tikzpicture}
		\caption{
		Hue (Pink)
		}
		\label{fig:colormapping:intro:differntColorMaps:hue}
	\end{subfigure}
	\begin{subfigure}{0.35\textwidth}
		\centering
		\includegraphics[width=0.9\textwidth, trim={35px 30px 430px 30px}, clip]{colormapping/img/heat}
		\begin{tikzpicture}
		    \node[anchor=south west,inner sep=0] (image) at (0,0) {\includegraphics[rotate=90,width=0.03\textwidth,height=95px,keepaspectratio=false,frame]{colormapping/img/colormap_legends/heatcolormap}};
		\end{tikzpicture}
		\caption{
		Heat
		}
		\label{fig:colormapping:intro:differntColorMaps:heat}
	\end{subfigure}
		\hspace{30px}
	\begin{subfigure}{0.35\textwidth}
		\centering
		\includegraphics[width=0.9\textwidth, trim={35px 30px 430px 30px}, clip]{colormapping/img/cold}
				\begin{tikzpicture}
		    \node[anchor=south west,inner sep=0] (image) at (0,0) {\includegraphics[rotate=90,width=0.03\textwidth,height=95px,keepaspectratio=false,frame]{colormapping/img/colormap_legends/coldcolormap}};
		\end{tikzpicture}
		\caption{
		Cold
		}
		\label{fig:colormapping:intro:differntColorMaps:cold}
	\end{subfigure}	
	\begin{subfigure}{0.35\textwidth}
		\centering
		\includegraphics[width=0.9\textwidth, trim={35px 30px 430px 30px}, clip]{colormapping/img/twocolors}
		\begin{tikzpicture}
		    \node[anchor=south west,inner sep=0] (image) at (0,0) {\includegraphics[rotate=90,width=0.03\textwidth,height=95px,keepaspectratio=false,frame]{colormapping/img/colormap_legends/twocolorscolormap}};
		\end{tikzpicture}
		\caption{
		Isoluminant Blue-Red
		}
		\label{fig:colormapping:intro:differntColorMaps:twocolor}
	\end{subfigure}	
	\hspace{30px}
	\begin{subfigure}{0.35\textwidth}
		\centering
		\includegraphics[width=0.9\textwidth, trim={35px 30px 430px 30px}, clip]{colormapping/img/diverging}
		\begin{tikzpicture}
		    \node[anchor=south west,inner sep=0] (image) at (0,0) {\includegraphics[rotate=90,width=0.03\textwidth,height=95px,keepaspectratio=false,frame]{colormapping/img/colormap_legends/divergingcolormap}};
		\end{tikzpicture}
		\caption{
		Diverging
		}
		\label{fig:colormapping:intro:differntColorMaps:diverging}
	\end{subfigure}		
\caption{A visualization of the fluid density (\density) using the color maps available in the application. All color maps uses 256 colors except for the zebra color map which uses 50. All color maps are fully saturated.}
\label{fig:colormapping:colormaps}
\end{figure}


In \cref{fig:colormapping:colormaps} a visualization from simulation snapshot is shown using all the different color maps available in the application. Comparing these figures we observe firstly that in the visualization with the rainbow color map, \cref{fig:colormapping:intro:differntColorMaps:rainbow}, the maximum values are very prominent and that there is a clear distinction between the blue, green, and green areas but no clear transition between those areas. This fits with the disadvantages discussed in \cref{ssub:rainbow_colormap}. \Cref{fig:colormapping:intro:differntColorMaps:zebra} indeed illustrates the areas with high and low variation. 

Comparing the luminance based color maps, \crefrange{fig:colormapping:intro:differntColorMaps:grayscale}{fig:colormapping:intro:differntColorMaps:cold}, we observe in all cases  a nice transition and from areas with low to areas with high scalar values. Also note the added benefit of the maps containing hues compared to the gray-scale color map; especially in the areas containing low values a bit more variation is visible.


The isoluminant blue-red color map, shown in \cref{fig:colormapping:intro:differntColorMaps:twocolor}, does not offer as much details as the other color maps. This confirms that these color maps are not well suited for 2D visualizations.

In the diverging color map, illustrated in \cref{fig:colormapping:intro:differntColorMaps:diverging}, one easily recognizes the same distinct areas shown by the rainbow color map.  Compared to the rainbow color map, there are two distinct differences. First the maximum values do not draw as much attention as they do in the rainbow color map, furthermore we can see a more natural transition from low to high values making this color map more suitable for visualization of these data.

% chapter streamlines (end)