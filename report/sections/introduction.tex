%!TEX root = ../scivis_lbaakman_bvanloon.tex
\chapter*{Introduction} % (fold)
\label{cha:introduction}
\addcontentsline{toc}{chapter}{Introduction} \markboth{INTRODUCTION}{}
This document accompanies the visualization application made for the course Scientific Visualization at the University of Groningen. The application provides different methods and engines to visualize a real-time fluid flow simulation. This simulation calculates the values of quantities such as fluid density and fluid velocity over time. In \cref{cha:skeleton_compilation} the simulation is shortly discussed as well as the techniques used to develop the simulation. Next \cref{cha:color_mapping} discusses the different colormaps available in the application. Vector visualization techniques are given in \cref{cha:glyphs} which introduces glyphs as a way to visualize vector data. \Cref{cha:gradients} shows how the gradient vector is calculated and can be used. Next \cref{cha:streamlines} introduces streamlines with which the stream of a seed-point is followed. \Cref{cha:slices} and \cref{cha:stream_surfaces} introduce two 3D visualization which can be used to show the visualization as a function over time.

% chapter introduction (end)