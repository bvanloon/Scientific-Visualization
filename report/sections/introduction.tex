%!TEX root = ../scivis_lbaakman_bvanloon.tex
\chapter*{Introduction} % (fold)
\label{cha:introduction}
\addcontentsline{toc}{chapter}{Introduction} \markboth{INTRODUCTION}{}
%
This document accompanies the visualization application made for the course Scientific Visualization at the University of Groningen. The application provides different methods to visualize a real-time fluid flow simulation. 
This simulation calculates the values of quantities such as fluid density and fluid velocity over time. In \cref{cha:skeleton_compilation} the simulation is shortly discussed as well as the techniques used to develop the simulation. 

Next \cref{cha:color_mapping} focuses on the visualization of scalars with colors. \Cref{cha:glyphs} and \cref{cha:streamlines} are concerned with the visualization of time-independent vector fields, by way of glyphs and streamlines, respectively. In \cref{cha:gradients} new vector fields, namely the gradient fields are introduced, and visualized with glyphs. 

Lastly \cref{cha:slices} and \cref{cha:stream_surfaces} discuss the visualization of time-dependent vector fields, with slices and stream surfaces, respectively. 

% chapter introduction (end)